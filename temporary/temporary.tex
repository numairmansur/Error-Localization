\documentclass{article}
\usepackage{blindtext}
\usepackage{listings}
\usepackage{color}
\usepackage{amsmath}
\usepackage{amsthm}
\usepackage{graphicx}
\usepackage{amsmath,amssymb,amsthm}
\newcommand{\limp}{\Rightarrow}
\newcommand{\WP}[2]{\mathit{WP}(#1,#2)}
\newcommand{\SP}[2]{\mathit{SP}(#1,#2)}
\newcommand{\havoc}{\mathit{havoc}}
\newcommand{\guard}{\mathit{guard}}
\definecolor{codegreen}{rgb}{0,0.6,0}
\definecolor{codegray}{rgb}{0.5,0.5,0.5}
\definecolor{codepurple}{rgb}{0.58,0,0.82}
\definecolor{backcolour}{rgb}{0.95,0.95,0.92} 
\lstdefinestyle{mystyle}{
    backgroundcolor=\color{backcolour},   
    commentstyle=\color{codegreen},
    keywordstyle=\color{magenta},
    numberstyle=\tiny\color{codegray},
    stringstyle=\color{codepurple},
    basicstyle=\footnotesize,
    breakatwhitespace=false,         
    breaklines=true,                 
    captionpos=b,                    
    keepspaces=true,                 
    numbers=left,                    
    numbersep=5pt,                  
    showspaces=false,                
    showstringspaces=false,
    showtabs=false,                  
    tabsize=2
}
\newtheorem{mydef}{Definition}
\newtheorem{theorem}{Theorem}
\newtheorem{lemma}{Lemma}

\lstset{style=mystyle}
\usepackage[utf8]{inputenc}
\title{Fault Localization \& Relevance Analysis \\ }
\author{Matthias Heizmann, Christian Schilling, Numair Mansur}

\begin{document}

\maketitle
\begin{mydef}[Execution]\label{mydef:execution_definition}
Let $\pi$ be an error trace of length $n$. An execution of $\pi$ is a sequence of states $s_0, s_1...s_n$ such that $s_i, s_{i+1} \models T$, where $T$ is the transition formula of $\pi[i]$. \\
Let $\epsilon$ represent the set of all possible executions of the error trace.
\end{mydef}

\begin{mydef}[Blocking Execution]\label{mydef:blockingexecution_definition}
An execution of a trace $\pi$ of size $n$ is called a blocking execution if there exists a sequence of states $s_0, s_1...s_j$ where $i<j \leq n$ such that $s_i, s_{i+1} \models T[i]$, where $T[i]$ is the transition formula of $\pi[i]$ and there exits an assume statement in the trace $\pi$ at position $j$ such that $s_{j} \not \limp guard(\pi[j])$.
\end{mydef}

\begin{mydef}[Relevancy of an assignment statement]\label{mydef:relevancy_definition}
Let $\beta$ represent the set of all blocking executions of a trace $\pi$. Let there be an assignment statement of the form $x:=t$ at position $i$. Let $\pi'$ represent the trace that we get after replacing $\pi[i]$ with a havoc statement of the form $havoc(x)$ and let $\beta'$ represent the set of all blocking executions for $\pi'$.\\
We say that the assignment statement $\pi[i]$ is relevant if the trace after the replacement has strictly more blocked executions than the trace before the replacement, i.e if $\beta \subsetneq \beta'$. 
\end{mydef}
\newpage

\begin{theorem}[Relevancy of an assignment statement]\label{mydef:relevancytheorem}
Let $\pi$ be an error trace of length $n$ and $\pi[i]$ be an assignment statement at position $i$ having the form $x:=t$, where $x$ is a variable and $t$ is an expression. Let $P$ and $Q$ be two predicates where $P = \neg WP(False; \pi[i,n]) \bigcap SP(True; \pi[1, i-1])$ and $Q =  \neg WP(False; \pi[i+1,n])$. The statement $\pi[i]$ is relevant iff:
 $$P \not \limp WP(Q,havoc(x))$$
\end{theorem}

\begin{proof}
Let $\pi'$ be the trace where the assignment statement $\pi[i]$ is replaced by a havoc statement. \\
"$\Rightarrow$" \\
If the assignment statement $\pi[i]$ is relevant then:
$$P \not \limp WP(Q,havoc(x))$$
The relevancy of the assignment statement $\pi[i]$ implies that the trace $\pi'$ have strictly more blocking exections then $\pi$. This means that there exists an assume statement in the trace $\pi$ at position $j$, which is blocking more executions then before. Or we can say that there are more states $s_j$ for which the assume statement is blocking. \\
OR
something something.
\end{proof}
\end{document}