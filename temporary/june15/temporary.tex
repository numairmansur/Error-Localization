\documentclass{article}
\usepackage{authblk}
\usepackage{blindtext}
\usepackage{listings}
\usepackage{color}
\usepackage{amsmath}
\usepackage{amsthm}
\usepackage{graphicx}
\usepackage{amsmath,amssymb,amsthm}
\newcommand{\limp}{\Rightarrow}
\newcommand{\WP}[2]{\mathit{WP}(#1,#2)}
\newcommand{\SP}[2]{\mathit{SP}(#1,#2)}
\newcommand{\havoc}{\mathit{havoc}}
\newcommand{\guard}{\mathit{guard}}
\definecolor{codegreen}{rgb}{0,0.6,0}
\definecolor{codegray}{rgb}{0.5,0.5,0.5}
\definecolor{codepurple}{rgb}{0.58,0,0.82}
\definecolor{backcolour}{rgb}{0.95,0.95,0.92} 
\lstdefinestyle{mystyle}{
    backgroundcolor=\color{backcolour},   
    commentstyle=\color{codegreen},
    keywordstyle=\color{magenta},
    numberstyle=\tiny\color{codegray},
    stringstyle=\color{codepurple},
    basicstyle=\footnotesize,
    breakatwhitespace=false,         
    breaklines=true,                 
    captionpos=b,                    
    keepspaces=true,                 
    numbers=left,                    
    numbersep=5pt,                  
    showspaces=false,                
    showstringspaces=false,
    showtabs=false,                  
    tabsize=2
}
\newtheorem{mydef}{Definition}
\newtheorem{theorem}{Theorem}
\newtheorem{lemma}{Lemma}

\lstset{style=mystyle}
\usepackage[utf8]{inputenc}
\title{Error Localization \& Relevance Analysis \\ }
\author{Matthias Heizmann, Christian Schilling, Numair Mansur}
\affil{University of Freiburg, Germany}
\date{\vspace{-5ex}}
\begin{document}

\begin{mydef}[Execution]\label{mydef:execution}
Let $\pi$ be an error trace of length $n$. An execution of $\pi$ is a sequence of states $s_0, s_1 ... s_n$ such that $s_i, s_{i+1} \vDash T$, where $T$ is the transition formula of $\pi[i]$.
\end{mydef}
\begin{mydef}[Blocked Execution]\label{mydef:blocked_execution}
An execution of a trace $\pi$ of size $n$ is called a blocked execution, if there exists a sequence of states $s_0,s_1...s_j$ where $i<j \leq n$ such that $s_i, s_{i+1} \vDash T$ where $T$ is the transition formula of $\pi[i]$ and there exists an assume statement in the trace $\pi$ at position $j$ such that $s_j \not \limp guard(\pi[j])$
\end{mydef}

\begin{mydef}[Relevant Statement]\label{mydef:responsible}
Let $\pi$ be an error trace of length $n$. Let there be an assignment statement at position $i$ of the form $x:=t$ where $x$ is a variable and $t$ is an expression. Let $P$ and $Q$ be two predicates such that for all possible executions of the trace $\pi$ with $s_i,s_{i+1} \vDash T$ , $s_i \in P$ and $s_{i+1} \in Q$.\\
The assignment statement $\pi[i]$ is relevant if we replace it with a havoc statement of the form $havoc(x)$ to get a new trace $\pi'$ and there exists a blocked execution with $s'_i,s'_{i+1} \vDash T'$ such that $T'$ is the transition formula for $havoc(x)$, $s'_i \in P$, $s'_{i+1} \in Q' \setminus Q$ where $Q \subsetneq Q'$.
\end{mydef}
\end{document}