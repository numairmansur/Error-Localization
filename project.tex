\documentclass{article}
\usepackage{blindtext}
\usepackage[utf8]{inputenc}

\title{Fault Localization}
\author{Numair Mansur}

\begin{document}
\maketitle
\section{Abstract}
The most time consuming part in a programmer’s routine is to spend time on debugging and to determine the cause of the error and what statements are actually responsible for the error. This is called \textbf{“Fault Localization”} . 
\\
Fault Localization encompasses the task of identification of the program statements that are \textbf{relevant}  for the error trace and determining the variables whose values should be tracked in order to understand the cause of the error [1]. We need to formally define what does it mean that a statement is relevant for an error.

\\
\\
\\
\\
\\
\\
\\
\\
\\
\\
\\
In order to add vertical space you have to use ``vspace''; for example, 
you could add an inch of space by typing \verb|\vspace{1in}|, like this:
\vspace{1in}

To get three lines of space you would type \verb|\vspace{3pc}|
(``pc'' stands for ``pica'', a font-relative size), like this:
\vspace{3pc}

Notice that \LaTeX\ commands are always preceeded by a backslash.  
Some commands, like \verb|\vspace|, take arguments (here, a length) in
curly brackets.  

The second important thing to notice about \LaTeX\ is that you type 
in various ``environments''...so far we've just been typing regular 
text (except for a few inescapable usages of \verb|\verb| and the
centered, smallcaps, large title).  There are basically two ways that 
you can enter and/or exit an environment;
\vspace{1pc}

\centerline{this is the first way...}

\begin{center}
this is the second way.
\end{center}

\noindent Actually there is one more way, used above; for example, 
{\sc this way}.  The way that you get in and out of environment varies
depending on which kind of environment you want; for example, you use 
\verb|\underline| ``outside'', but \verb|\it| ``inside''; 
notice \underline{this} versus {\it this}.

The real power of \LaTeX\ (for us) is in the math environment. You 
push and pop out of the math environment by typing \verb|$|. For 
example, $2x^3 - 1 = 5$ is typed between dollar signs as
\verb|$2x^3 - 1 = 5$|. Perhaps a more interesting example is
$\lim_{N \to \infty} \sum_{k=1}^N f(t_k) \Delta t$.

You can get a fancier, display-style math 
environment by enclosing your equation with double dollar signs.  
This will center your equation, and display sub- and super-scripts in 
a more readable fashion:

$$\lim_{N \to \infty} \sum_{k=1}^N f(t_k) \Delta t.$$

If you don't want your equation to be centered, but you want the nice 
indicies and all that, you can use \verb|\displaystyle| and get your 
formula ``in-line''; using our example this is 
$\displaystyle \lim_{N \to \infty} \sum_{k=1}^N f(t_k) \Delta t.$  Of 
course this can screw up your line spacing a little bit.

There are many more things to know about \LaTeX\ and we can't 
possibly talk about them all here.
You can use \LaTeX\ to get tables, commutative diagrams, figures, 
aligned equations, cross-references, labels, matrices, and all manner 
of strange things into your documents.  You can control margins, 
spacing, alignment, {\it et cetera} to higher degrees of accuracy than 
the human eye can percieve.  You can waste entire days typesetting 
documents to be ``just so''.  In short, \LaTeX\ rules.

The best way to learn \LaTeX\ is by example. Get yourself a bunch
of .tex files, see what kind of output they produce, and figure out how
to modify them to do what you want.  There are many template and 
sample files on the department \LaTeX\ page and in real life in the 
big binder that should be in the computer lab somewhere.  Good luck!











\end{document}