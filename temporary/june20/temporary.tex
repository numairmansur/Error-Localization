\documentclass{article}
\usepackage{authblk}
\usepackage{blindtext}
\usepackage{listings}
\usepackage{color}
\usepackage{amsmath}
\usepackage{amsthm}
\usepackage{graphicx}
\usepackage{amsmath,amssymb,amsthm}
\newcommand{\limp}{\Rightarrow}
\newcommand{\WP}[2]{\mathit{WP}(#1,#2)}
\newcommand{\SP}[2]{\mathit{SP}(#1,#2)}
\newcommand{\havoc}{\mathit{havoc}}
\newcommand{\guard}{\mathit{guard}}
\definecolor{codegreen}{rgb}{0,0.6,0}
\definecolor{codegray}{rgb}{0.5,0.5,0.5}
\definecolor{codepurple}{rgb}{0.58,0,0.82}
\definecolor{backcolour}{rgb}{0.95,0.95,0.92} 
\lstdefinestyle{mystyle}{
    backgroundcolor=\color{backcolour},   
    commentstyle=\color{codegreen},
    keywordstyle=\color{magenta},
    numberstyle=\tiny\color{codegray},
    stringstyle=\color{codepurple},
    basicstyle=\footnotesize,
    breakatwhitespace=false,         
    breaklines=true,                 
    captionpos=b,                    
    keepspaces=true,                 
    numbers=left,                    
    numbersep=5pt,                  
    showspaces=false,                
    showstringspaces=false,
    showtabs=false,                  
    tabsize=2
}
\newtheorem{mydef}{Definition}
\newtheorem{theorem}{Theorem}
\newtheorem{lemma}{Lemma}

\lstset{style=mystyle}
\usepackage[utf8]{inputenc}
\title{Error Localization \& Relevance Analysis \\ }
\author{Matthias Heizmann, Christian Schilling, Numair Mansur}
\affil{University of Freiburg, Germany}
\date{\vspace{-5ex}}
\begin{document}

\begin{mydef}[Execution]\label{mydef:execution}
Let $\pi$ be an error trace of length $n$. An execution of $\pi$ is a sequence of states $s_0, s_1 ... s_n$ such that $s_i, s_{i+1} \vDash T$, where $T$ is the transition formula of $\pi[i]$.
\end{mydef}
\begin{mydef}[Blocked Execution]\label{mydef:blocked_execution}
An execution of a trace $\pi$ of size $n$ is called a blocked execution, if there exists a sequence of states $s_0,s_1...s_j$ where $i<j \leq n$ such that $s_i, s_{i+1} \vDash T$ where $T$ is the transition formula of $\pi[i]$ and there exists an assume statement in the trace $\pi$ at position $j$ such that $s_j \not \limp guard(\pi[j])$
\end{mydef}

\begin{mydef}[Relevant Statement]\label{mydef:responsible}
Let $\pi = st_1,....,st_n$ be an error trace of length $n$ where $st_i$ is an assignment statement of the form $x:=t$. The assignemnt statement at position $i$ is relevant if there exists an execution $s_1,...s_{n+1}$ of $\pi$ and some value $v$ such that every execution of the trace $x:=v; \pi[i+1,n]$ starting in $s_i$ is has a blocked execution.
\end{mydef}

\begin{lemma}\label{lemma:duality}
For a program statement $st$ and predicates $P$ and $Q$, where $P$ is condition that is true before the execution of the statement and $Q$ is a post condition, the following two implications are equivilant(also known as the duality of WP and SP):
$$SP(P,st) \Rightarrow Q$$
$$P \Rightarrow WP(Q,st)$$
\end{lemma}

\begin{lemma}[]\label{lemma:rel_bw_assignment_and_havoc}
For a predicate $Q$ and an assignment statement of the form $x:=t$ where $x$ is a variable and $t$ is an expression, we have:
$$WP(Q; havoc(x)) \subseteq WP(Q; x:=t)$$
and
$$SP(P;x:=t) \subseteq SP(P;havoc(x))$$
\end{lemma}

\begin{lemma}[IGNORE FOR NOW]\label{lemma:nonempty_post}
	For $P := \WP{Q}{x := t}$ and a set of states $R$, if $ P \cap R \nsubseteq \WP{Q}{\havoc(x)}$ for some $Q$ then $Q \subsetneq \SP{P}{\havoc(x)}$.
\end{lemma}
\begin{proof}
	We will show that $Q : =SP(P; x:=t) \subseteq SP(P;havoc(x)) \not\subseteq SP(P; x:=t)$ from which it follows that the first inclusion is strict.
	The first inclusion is from Lemma~\ref{lemma:rel_bw_assignment_and_havoc}. It is obvious that a state reachable after $x:=t$ is also reachable 
	after $havoc(x)$. Hence $SP(P; x:=t) \subseteq SP(P;havoc(x))$. \\
	By assumption $WP(Q;x:=t) \cap R \not\subseteq \WP{Q}{\havoc(x)}$, which is equivalent to $WP(Q; x:=t) \not \subseteq \ WP(Q; havoc(x))$ which by Lemma~\ref{lemma:duality} is equivalent to.
	$$SP(WP(Q;x:=t); havoc(x)) \not\subseteq Q$$
	or 
	$$SP(P;havoc(x)) \not\subseteq SP(P; x:=t)$$
\end{proof}

\begin{theorem}[Relevancy of an assignment statement]\label{mydef:relevancytheorem}
Let $\pi$ be an error trace of length $n$ and $\pi[i]$ be an assignment statement at position $i$ having the form $x:=t$, where $x$ is a variable and $t$ is an expression. Let $P$ and $Q$ be two predicates where $P = \neg WP(False; \pi[i,n]) \cap SP(True; \pi[1,i-1])$ and $Q =  \neg WP(False; \pi[i+1,n])$. The statement $\pi[i]$ is relevant iff:
 $$P \not \limp WP(Q,havoc(x))$$
\end{theorem}

\begin{proof}
Let $P' = WP(Q;havoc(x)) \cap SP(True; \pi[1,i-1])$ and $Q'=SP(P;havoc(x))$. It is obvious that $P$ can also be written as $WP(Q;x:=t) \cap SP(True; \pi[1,i-1])$ and $Q$ as $SP(P;x:=t)$.\\
"$\Rightarrow$"\\
If $\pi[i]$ is relevant, then\\
$$P \not \limp WP(Q;havoc(x))$$
Obviously all the transition from $P'$ end up in $Q$. Relevancy of $x:=t$ implies that there is a state in $s \in P$ such that there is a transition from $s$ to $\neg Q$. That would mean:
$$P \not \limp P'$$
$$P \not \limp WP(Q;havoc(x))$$
"$\Leftarrow$"\\
$\pi[i]$ is relevant, if:
$$P \not \limp WP(Q;havoc(x))$$
From lemma~\ref{lemma:duality}, we can write:
$$SP(P; havoc(x)) \not \limp Q$$
$$Q' \not \limp Q$$
This shows the existence of a state $s$ in $Q'$ such that $s\in \neg Q$ and hence a value $v$ for $x$ such that if we replace $x:=t$ with $x:=v$, then every execution is becoming blocking. Also, from our assumption, it is clear that there exists an execution till $P$, since $P$ is not empty.


\end{proof}
\end{document}