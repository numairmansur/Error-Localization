\documentclass{article}
\usepackage{authblk}
\usepackage{blindtext}
\usepackage{listings}
\usepackage{color}
\usepackage{amsmath}
\usepackage{amsthm}
\usepackage{graphicx}
\usepackage{amsmath,amssymb,amsthm}
\usepackage{algorithm}
\usepackage[noend]{algpseudocode}
\newcommand{\limp}{\Rightarrow}
\newcommand{\WP}[2]{\mathit{WP}(#1,#2)}
\newcommand{\SP}[2]{\mathit{SP}(#1,#2)}
\newcommand{\havoc}{\mathit{havoc}}
\newcommand{\guard}{\mathit{guard}}
\definecolor{codegreen}{rgb}{0,0.6,0}
\definecolor{codegray}{rgb}{0.5,0.5,0.5}
\definecolor{codepurple}{rgb}{0.58,0,0.82}
\definecolor{backcolour}{rgb}{0.95,0.95,0.92} 
\lstdefinestyle{mystyle}{
    backgroundcolor=\color{backcolour},   
    commentstyle=\color{codegreen},
    keywordstyle=\color{magenta},
    numberstyle=\tiny\color{codegray},
    stringstyle=\color{codepurple},
    basicstyle=\footnotesize,
    breakatwhitespace=false,         
    breaklines=true,                 
    captionpos=b,                    
    keepspaces=true,                 
    numbers=left,                    
    numbersep=5pt,                  
    showspaces=false,                
    showstringspaces=false,
    showtabs=false,                  
    tabsize=2
}
\newtheorem{mydef}{Definition}
\newtheorem{theorem}{Theorem}
\newtheorem{lemma}{Lemma}

\lstset{style=mystyle}
\usepackage[utf8]{inputenc}
\title{Error Localization \& Relevance Analysis \\ }
\author{Matthias Heizmann, Christian Schilling, Numair Mansur}
\affil{University of Freiburg, Germany}
\date{\vspace{-5ex}}
\begin{document}
% ******************************************** BEGIN DOCUMENT **************************************************************************
\begin{mydef}[Execution]\label{mydef:execution}
Let $\pi$ be an error trace of length $n$. An execution of $\pi$ is a sequence of states $s_0, s_1 ... s_n$ such that $s_i, s_{i+1} \vDash T$, where $T$ is the transition formula of $\pi[i]$.
\end{mydef}
\begin{mydef}[Blocking Execution]\label{mydef:blocked_execution}
An execution of a trace $\pi$ of size $n$ is called a blocking execution, if there exists a sequence of states $s_0,s_1...s_j$ where $i<j \leq n$ such that $s_i, s_{i+1} \vDash T$ where $T$ is the transition formula of $\pi[i]$ and there exists an assume statement in the trace $\pi$ at position $j$ such that $s_j \not \limp guard(\pi[j])$
\end{mydef}

\begin{mydef}[Relevance of an assigning statement]\label{mydef:responsible}
Let $\pi = \langle st_1,....,st_n \rangle$ be an error trace of length $n$ where $st_i$ is an assigning statement at position $i$ that assigns a new value to some variable $x$. The statement $st_i$ is relevant if there exists an execution $s_1,...s_{n+1}$ of $\pi$ and some value $v$ such that every execution of the trace $\langle x:=v; \pi[i+1,n] \rangle$ starting in $s_i$ has a blocking execution.
\end{mydef}

\newpage
% ******************************************** NEW PAGE *******************************************************************************
% Checkout the following link for writing an algo in Latex
% https://tex.stackexchange.com/questions/163768/write-pseudo-code-in-latex
\begin{algorithm}
\caption{Relavance of an assigning statement}\label{relevance}
\begin{algorithmic}[1]
\Procedure{Relevance}{}
\State write me
\EndProcedure
\end{algorithmic}
\end{algorithm}
In the algorithm , we check the relevance of a statement by checking if the triple ($P,\pi[i],\neg Q$) is unsatisfiable and $\pi[i]$ is in the unsatisfiable core. We can do this by checking if $P \not \subseteq WP(Q; havoc(x))$ .


\newpage
% ******************************************** NEW PAGE *******************************************************************************
\begin{theorem}[Equivalence of relevance]\label{mydef:relevancytheorem}
Let $\pi = \langle st_1,...,st_i,...,st_n \rangle$ be an error trace of length $n$ and $\pi[i]$ be an assigning statement at position $i$ , which assigns a new value to some variable $x$ . Let $P = \neg WP(False; \pi[i,n]) \cap SP(True; \pi[1,i-1])$ be a set of bireachable states at position $i$ and $Q =  \neg WP(False; \pi[i+1,n])$ be the coreachable states at position $i+1$. The statement $\pi[i]$ is relevant iff:
 $$P \not \subseteq WP(Q,havoc(x))$$
\end{theorem}

\begin{proof}
Let $\mathcal{D}$ be the domain of the variable $x$. \\
"$\Rightarrow$"\\
If $\pi[i]$ is relevant, then\\
$$P \not \subseteq WP(Q;havoc(x))$$
Obviously all the transitions from the states in $WP(Q;havoc(x))$ ends up in $Q$. Relevancy of $\pi[i]$ implies that there is a state in $s \in P$ such that there is a transition from $s$ to $\neg Q$. That would mean:
$$P \not \subseteq WP(Q;havoc(x))$$
"$\Leftarrow$"\\
$\pi[i]$ is relevant, if:
$$P \not \subseteq WP(Q;havoc(x))$$
We know that $WP(Q;havoc(x))$ is the set of states from which all transitions end up in $Q$. The above non implication shows the existence of a state $s$ in $P$ such that $s \not \in WP(Q; havoc(x))$  from which there is a transition to $\neg Q$. This shows the existence of a value $v \in \mathcal{D}$ that we can assign to $x$ such that if we replace $\pi[i]$ with $x:=v$, then every execution is becoming blocking. Also, from our assumption, it is clear that there exits an execution till $P$, since $P$ is not empty.
\end{proof}


\end{document}